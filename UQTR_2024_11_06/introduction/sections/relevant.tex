%-------------------
% PERTINENCE GLOBALE
%-------------------

% Defining TOC's sections before content
\section{Pourquoi \LaTeX\;est-il pertinent~?}

\begin{frame}{Pourquoi \LaTeX\;est-il pertinent~?}
    La pertinence de ce langage peut être résumée en 5 points~:
    \vspace{0.5cm}
    \pause
    \begin{enumerate}
        \item Qualité typographique et exigences mathématiques
        \pause
    \end{enumerate}
    \begin{align}
        \exp(x) = \faxoperator_{n=0}^\infty\frac{x^n}{n!}
        \label{eq: faxop_exp}
    \end{align}
\end{frame}

\begin{frame}{Pourquoi \LaTeX\;est-il pertinent~?}
    La pertinence de ce langage peut être résumée en 5 points~:
    \vspace{0.5cm}
    \begin{enumerate}
        \item Qualité typographique et exigences mathématiques
        \item Contrôle précis de la mise en page (\cancel{WYSIWYG})
        \pause
        \item Alignement des figures stable, professionnel et sans
        perte de qualité
        \pause
        \item Standard de publication pour de nombreuses revues académiques
        \pause
        \item Automatisation (réduction des erreurs de références, numérotation, etc.)
    \end{enumerate}
    \vspace{0.5cm}
    \pause
    et surtout...
\end{frame}

\begin{frame}{Pourquoi \LaTeX\;est-il pertinent~?}
    La pertinence de ce langage peut être résumée en 5 points~:
    \vfill
    \begin{enumerate}
        \item Qualité typographique et exigences mathématiques
        \item Contrôle précis de la mise en page (\cancel{WYSIWYG})
        \item Alignement des figures stable, professionnel et sans
        perte de qualité
        \item Standard de publication pour de nombreuses revues académiques
        \item Automatisation (réduction des erreurs de références, numérotation, etc.)
    \end{enumerate}
    \vfill
    \centering
    \begin{mylittlegreenbox}
        \Large\PointingHand\;open source et rétrocompatible
    \end{mylittlegreenbox}
\end{frame}
